\section{Problem Findings}

\subsection{Inefficient Document Management}
Many organizations face challenges with managing large volumes of documents, leading to inefficiencies and increased operational costs. Research indicates that document management issues can lead to a productivity loss of approximately 21.3\%, costing businesses around ``\$19,732`` per information worker annually (IDC, 2023). A study published in the *International Journal of Professional Business Review* found that organizations with poor document management systems experience productivity losses of up to 20\% due to time spent searching for information (Malekany, 2023).

\subsection{Limited Accessibility}
Existing platforms often lack comprehensive accessibility features, making it difficult for users with disabilities to effectively utilize document management tools. A study in the *International Journal of Environmental Research and Public Health* emphasizes that about 70\% of digital content is inaccessible to people with disabilities, highlighting the urgent need for more inclusive solutions (Kiepek et al., 2022).

\subsection{Data Privacy and Security Concerns}
With the rise of data breaches, ensuring the privacy and security of sensitive information has become a critical issue for businesses and individuals alike. The Office of the Privacy Commissioner for Personal Data (PCPD) in Hong Kong reported a significant increase in data breaches, indicating that public sector organizations are particularly vulnerable (PCPD, 2023). Additionally, a study highlighted that organizations face fines or litigation due to poor document management practices, with 35\% of organizations reporting such incidents (AIIM, 2023).

\subsection{High Operational Costs}
Manual document processing and outdated systems result in high costs and resource wastage. The *Journal of Business Research* outlines how intelligent document management systems can automate processes, potentially reducing operational overheads by up to 30\% (Valaitis et al., 2024). Furthermore, research shows that the average cost to manually manage paper documents is about ``\$20`` per document, which adds up significantly across an organization’s operations (PwC, 2023).

\subsection{Lack of Customization and Flexibility}
Many solutions offer limited customization options, failing to meet the diverse needs of different industries and organizations. A survey indicated that organizations adopting tailored document management solutions see enhanced performance and user satisfaction (Malekany, 2023).

\subsection{Environmental Impact}
Traditional document management relies heavily on paper, contributing to environmental degradation and unsustainable practices. According to Greenpeace, producing one ton of paper results in approximately 1.3 tons of CO2 emissions. Transitioning to digital document management can significantly reduce paper usage; an average office worker uses about 10,000 sheets of paper each year (Greenpeace, 2023).

\subsection{The Complexity and Technical Barriers}
Many form builders require users to have technical skills or coding knowledge to create effective forms. This complexity can deter non-technical users from utilizing these tools, leading to underutilization of digital data collection methods.

\subsection{Limited Customization Options}
Existing solutions often provide a rigid structure that does not allow for sufficient customization in terms of design and functionality. Users may struggle to create forms that align with their branding or specific needs, resulting in a poor user experience for respondents.

\subsection{Inefficient Data Collection Processes and Error Reduction}
Traditional methods of form creation can be time-consuming and cumbersome, especially when dealing with paper forms or static digital formats. This inefficiency can lead to delays in data collection and analysis. High rates of human error in data collection and processing can lead to significant inefficiencies and quality issues. Traditional methods often rely on manual input, which is prone to mistakes.

\subsection{Lack of Real-Time Feedback and Validation}
Many tools do not offer immediate validation of user inputs, which can result in errors and incomplete submissions. This lack of real-time feedback can frustrate users and lead to lower completion rates.

\subsection{Inadequate Reporting and Analytics}
Organizations often struggle with extracting insights from the data collected through forms due to limited reporting capabilities in existing tools. Without robust analytics features, it becomes challenging to make informed decisions based on the collected data.

\subsection{Insufficient Integration with Other Tools}
The inability to seamlessly integrate with other applications (such as CRM systems, email marketing platforms, database, etc.) limits the utility of the data collected through forms. This fragmentation can hinder workflow efficiency and data utilization.

\subsection{Time-consuming development cycles for creating and modifying forms}

\subsection{Lack of flexibility in form design and functionality}

\subsection{Lack of Real-Time Feedback and Validation}


